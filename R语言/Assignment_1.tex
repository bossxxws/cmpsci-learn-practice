% Options for packages loaded elsewhere
\PassOptionsToPackage{unicode}{hyperref}
\PassOptionsToPackage{hyphens}{url}
%
\documentclass[
]{article}
\usepackage{amsmath,amssymb}
\usepackage{iftex}
\ifPDFTeX
  \usepackage[T1]{fontenc}
  \usepackage[utf8]{inputenc}
  \usepackage{textcomp} % provide euro and other symbols
\else % if luatex or xetex
  \usepackage{unicode-math} % this also loads fontspec
  \defaultfontfeatures{Scale=MatchLowercase}
  \defaultfontfeatures[\rmfamily]{Ligatures=TeX,Scale=1}
\fi
\usepackage{lmodern}
\ifPDFTeX\else
  % xetex/luatex font selection
\fi
% Use upquote if available, for straight quotes in verbatim environments
\IfFileExists{upquote.sty}{\usepackage{upquote}}{}
\IfFileExists{microtype.sty}{% use microtype if available
  \usepackage[]{microtype}
  \UseMicrotypeSet[protrusion]{basicmath} % disable protrusion for tt fonts
}{}
\makeatletter
\@ifundefined{KOMAClassName}{% if non-KOMA class
  \IfFileExists{parskip.sty}{%
    \usepackage{parskip}
  }{% else
    \setlength{\parindent}{0pt}
    \setlength{\parskip}{6pt plus 2pt minus 1pt}}
}{% if KOMA class
  \KOMAoptions{parskip=half}}
\makeatother
\usepackage{xcolor}
\usepackage[margin=1in]{geometry}
\usepackage{color}
\usepackage{fancyvrb}
\newcommand{\VerbBar}{|}
\newcommand{\VERB}{\Verb[commandchars=\\\{\}]}
\DefineVerbatimEnvironment{Highlighting}{Verbatim}{commandchars=\\\{\}}
% Add ',fontsize=\small' for more characters per line
\usepackage{framed}
\definecolor{shadecolor}{RGB}{248,248,248}
\newenvironment{Shaded}{\begin{snugshade}}{\end{snugshade}}
\newcommand{\AlertTok}[1]{\textcolor[rgb]{0.94,0.16,0.16}{#1}}
\newcommand{\AnnotationTok}[1]{\textcolor[rgb]{0.56,0.35,0.01}{\textbf{\textit{#1}}}}
\newcommand{\AttributeTok}[1]{\textcolor[rgb]{0.13,0.29,0.53}{#1}}
\newcommand{\BaseNTok}[1]{\textcolor[rgb]{0.00,0.00,0.81}{#1}}
\newcommand{\BuiltInTok}[1]{#1}
\newcommand{\CharTok}[1]{\textcolor[rgb]{0.31,0.60,0.02}{#1}}
\newcommand{\CommentTok}[1]{\textcolor[rgb]{0.56,0.35,0.01}{\textit{#1}}}
\newcommand{\CommentVarTok}[1]{\textcolor[rgb]{0.56,0.35,0.01}{\textbf{\textit{#1}}}}
\newcommand{\ConstantTok}[1]{\textcolor[rgb]{0.56,0.35,0.01}{#1}}
\newcommand{\ControlFlowTok}[1]{\textcolor[rgb]{0.13,0.29,0.53}{\textbf{#1}}}
\newcommand{\DataTypeTok}[1]{\textcolor[rgb]{0.13,0.29,0.53}{#1}}
\newcommand{\DecValTok}[1]{\textcolor[rgb]{0.00,0.00,0.81}{#1}}
\newcommand{\DocumentationTok}[1]{\textcolor[rgb]{0.56,0.35,0.01}{\textbf{\textit{#1}}}}
\newcommand{\ErrorTok}[1]{\textcolor[rgb]{0.64,0.00,0.00}{\textbf{#1}}}
\newcommand{\ExtensionTok}[1]{#1}
\newcommand{\FloatTok}[1]{\textcolor[rgb]{0.00,0.00,0.81}{#1}}
\newcommand{\FunctionTok}[1]{\textcolor[rgb]{0.13,0.29,0.53}{\textbf{#1}}}
\newcommand{\ImportTok}[1]{#1}
\newcommand{\InformationTok}[1]{\textcolor[rgb]{0.56,0.35,0.01}{\textbf{\textit{#1}}}}
\newcommand{\KeywordTok}[1]{\textcolor[rgb]{0.13,0.29,0.53}{\textbf{#1}}}
\newcommand{\NormalTok}[1]{#1}
\newcommand{\OperatorTok}[1]{\textcolor[rgb]{0.81,0.36,0.00}{\textbf{#1}}}
\newcommand{\OtherTok}[1]{\textcolor[rgb]{0.56,0.35,0.01}{#1}}
\newcommand{\PreprocessorTok}[1]{\textcolor[rgb]{0.56,0.35,0.01}{\textit{#1}}}
\newcommand{\RegionMarkerTok}[1]{#1}
\newcommand{\SpecialCharTok}[1]{\textcolor[rgb]{0.81,0.36,0.00}{\textbf{#1}}}
\newcommand{\SpecialStringTok}[1]{\textcolor[rgb]{0.31,0.60,0.02}{#1}}
\newcommand{\StringTok}[1]{\textcolor[rgb]{0.31,0.60,0.02}{#1}}
\newcommand{\VariableTok}[1]{\textcolor[rgb]{0.00,0.00,0.00}{#1}}
\newcommand{\VerbatimStringTok}[1]{\textcolor[rgb]{0.31,0.60,0.02}{#1}}
\newcommand{\WarningTok}[1]{\textcolor[rgb]{0.56,0.35,0.01}{\textbf{\textit{#1}}}}
\usepackage{graphicx}
\makeatletter
\def\maxwidth{\ifdim\Gin@nat@width>\linewidth\linewidth\else\Gin@nat@width\fi}
\def\maxheight{\ifdim\Gin@nat@height>\textheight\textheight\else\Gin@nat@height\fi}
\makeatother
% Scale images if necessary, so that they will not overflow the page
% margins by default, and it is still possible to overwrite the defaults
% using explicit options in \includegraphics[width, height, ...]{}
\setkeys{Gin}{width=\maxwidth,height=\maxheight,keepaspectratio}
% Set default figure placement to htbp
\makeatletter
\def\fps@figure{htbp}
\makeatother
\setlength{\emergencystretch}{3em} % prevent overfull lines
\providecommand{\tightlist}{%
  \setlength{\itemsep}{0pt}\setlength{\parskip}{0pt}}
\setcounter{secnumdepth}{-\maxdimen} % remove section numbering
\ifLuaTeX
  \usepackage{selnolig}  % disable illegal ligatures
\fi
\usepackage{bookmark}
\IfFileExists{xurl.sty}{\usepackage{xurl}}{} % add URL line breaks if available
\urlstyle{same}
\hypersetup{
  pdftitle={Assignment 1},
  pdfauthor={ZHANGAO 2022214514},
  hidelinks,
  pdfcreator={LaTeX via pandoc}}

\title{Assignment 1}
\author{ZHANGAO 2022214514}
\date{}

\begin{document}
\maketitle

\begin{verbatim}
## Warning: 程序包's20x'是用R版本4.4.1 来建造的
\end{verbatim}

\section{Question 1}\label{question-1}

\section{Question of interest/goal of the
study}\label{question-of-interestgoal-of-the-study}

We are interested in using a country's gross domestic product to predict
the amount of electricity that they use.

\section{Read in and inspect the
data:}\label{read-in-and-inspect-the-data}

\begin{Shaded}
\begin{Highlighting}[]
\NormalTok{elec.df}\OtherTok{\textless{}{-}}\FunctionTok{read.csv}\NormalTok{(}\StringTok{"electricity.csv"}\NormalTok{)}
\FunctionTok{plot}\NormalTok{(Electricity}\SpecialCharTok{\textasciitilde{}}\NormalTok{GDP, }\AttributeTok{data=}\NormalTok{elec.df)}
\end{Highlighting}
\end{Shaded}

\includegraphics{Assignment_1_files/figure-latex/unnamed-chunk-2-1.pdf}

Countries with higher GDP also have more electricity

\section{Fit an appropriate linear model, including model
checks.}\label{fit-an-appropriate-linear-model-including-model-checks.}

\begin{Shaded}
\begin{Highlighting}[]
\NormalTok{elecfit1}\OtherTok{=}\FunctionTok{lm}\NormalTok{(Electricity}\SpecialCharTok{\textasciitilde{}}\NormalTok{GDP,}\AttributeTok{data=}\NormalTok{elec.df)}
\FunctionTok{plot}\NormalTok{(elecfit1,}\AttributeTok{which=}\DecValTok{1}\NormalTok{)}
\end{Highlighting}
\end{Shaded}

\includegraphics{Assignment_1_files/figure-latex/unnamed-chunk-3-1.pdf}

\begin{Shaded}
\begin{Highlighting}[]
\FunctionTok{normcheck}\NormalTok{(elecfit1)}
\end{Highlighting}
\end{Shaded}

\includegraphics{Assignment_1_files/figure-latex/unnamed-chunk-3-2.pdf}

\begin{Shaded}
\begin{Highlighting}[]
\FunctionTok{cooks20x}\NormalTok{(elecfit1)}
\end{Highlighting}
\end{Shaded}

\includegraphics{Assignment_1_files/figure-latex/unnamed-chunk-3-3.pdf}

\section{Identify the two countries with GDP greater than
6000.}\label{identify-the-two-countries-with-gdp-greater-than-6000.}

\begin{Shaded}
\begin{Highlighting}[]
\CommentTok{\# could use some R code to do this}
\CommentTok{\# Load the data}
\NormalTok{data }\OtherTok{\textless{}{-}} \FunctionTok{read.csv}\NormalTok{(}\StringTok{"electricity.csv"}\NormalTok{, }\AttributeTok{stringsAsFactors =} \ConstantTok{FALSE}\NormalTok{)}

\CommentTok{\# Filter the data for countries with GDP greater than 6000}
\NormalTok{high\_gdp\_countries }\OtherTok{\textless{}{-}} \FunctionTok{subset}\NormalTok{(data, GDP }\SpecialCharTok{\textgreater{}} \DecValTok{6000}\NormalTok{)}

\CommentTok{\# Print the results}
\FunctionTok{print}\NormalTok{(high\_gdp\_countries)}
\end{Highlighting}
\end{Shaded}

\begin{verbatim}
##         Country Electricity   GDP
## 4         China        3438  9872
## 27 UnitedStates        3873 14720
\end{verbatim}

The output is :

4 China 3438 9872

27 UnitedStates 3873 14720

So China and UnitedStates are the countries we want to find.

\section{Replot data eliminating countries with GDP greater than
6000.}\label{replot-data-eliminating-countries-with-gdp-greater-than-6000.}

\begin{Shaded}
\begin{Highlighting}[]
\CommentTok{\# Hint: If you want to limit the range of the data, do so in the data statement. E.G. something similar to data=elec.df[elec.df$GDP\textgreater{}2000,]}
\CommentTok{\# Load the necessary library}
\FunctionTok{library}\NormalTok{(ggplot2)}
\end{Highlighting}
\end{Shaded}

\begin{verbatim}
## Warning: 程序包'ggplot2'是用R版本4.4.1 来建造的
\end{verbatim}

\begin{Shaded}
\begin{Highlighting}[]
\CommentTok{\# Load the data}
\NormalTok{data }\OtherTok{\textless{}{-}} \FunctionTok{read.csv}\NormalTok{(}\StringTok{"electricity.csv"}\NormalTok{, }\AttributeTok{stringsAsFactors =} \ConstantTok{FALSE}\NormalTok{)}

\CommentTok{\# Filter the data to exclude countries with GDP greater than 6000}
\NormalTok{filtered\_data }\OtherTok{\textless{}{-}}\NormalTok{ data[data}\SpecialCharTok{$}\NormalTok{GDP }\SpecialCharTok{\textless{}=} \DecValTok{6000}\NormalTok{, ]}

\CommentTok{\# Create a scatter plot with the fitted line from the new fitted model superimposed}
\FunctionTok{ggplot}\NormalTok{(filtered\_data, }\FunctionTok{aes}\NormalTok{(}\AttributeTok{x =}\NormalTok{ GDP, }\AttributeTok{y =}\NormalTok{ Electricity)) }\SpecialCharTok{+}
  \FunctionTok{geom\_point}\NormalTok{() }\SpecialCharTok{+} \CommentTok{\# Add points to the plot}
  \FunctionTok{geom\_smooth}\NormalTok{(}\AttributeTok{method =} \StringTok{"lm"}\NormalTok{, }\AttributeTok{color =} \StringTok{"blue"}\NormalTok{) }\SpecialCharTok{+} \CommentTok{\# Add a linear regression line}
  \FunctionTok{labs}\NormalTok{(}\AttributeTok{title =} \StringTok{"Electricity Consumption vs GDP"}\NormalTok{,}
       \AttributeTok{x =} \StringTok{"GDP (billions of dollars)"}\NormalTok{,}
       \AttributeTok{y =} \StringTok{"Electricity Consumption (billions of kilowatt{-}hours)"}\NormalTok{) }\SpecialCharTok{+}
  \FunctionTok{theme\_minimal}\NormalTok{() }\CommentTok{\# Use a minimal theme for the plot}
\end{Highlighting}
\end{Shaded}

\begin{verbatim}
## `geom_smooth()` using formula = 'y ~ x'
\end{verbatim}

\includegraphics{Assignment_1_files/figure-latex/unnamed-chunk-5-1.pdf}

This is the graph eliminating countries with GDP greater than 6000.

\section{Fit a more appropriate linear model, including model
checks.}\label{fit-a-more-appropriate-linear-model-including-model-checks.}

\begin{Shaded}
\begin{Highlighting}[]
\CommentTok{\# Load the necessary library}
\FunctionTok{library}\NormalTok{(ggplot2)}
\FunctionTok{library}\NormalTok{(ggfortify) }\CommentTok{\# For autoplot function}
\end{Highlighting}
\end{Shaded}

\begin{verbatim}
## Warning: 程序包'ggfortify'是用R版本4.4.1 来建造的
\end{verbatim}

\begin{Shaded}
\begin{Highlighting}[]
\CommentTok{\# Load the data}
\NormalTok{electricity\_data }\OtherTok{\textless{}{-}} \FunctionTok{read.csv}\NormalTok{(}\StringTok{"electricity.csv"}\NormalTok{, }\AttributeTok{stringsAsFactors =} \ConstantTok{FALSE}\NormalTok{)}

\CommentTok{\# Filter the data to exclude countries with GDP greater than 6000}
\NormalTok{filtered\_data }\OtherTok{\textless{}{-}}\NormalTok{ electricity\_data[electricity\_data}\SpecialCharTok{$}\NormalTok{GDP }\SpecialCharTok{\textless{}=} \DecValTok{6000}\NormalTok{, ]}

\CommentTok{\# Fit a linear model}
\NormalTok{linear\_model }\OtherTok{\textless{}{-}} \FunctionTok{lm}\NormalTok{(Electricity }\SpecialCharTok{\textasciitilde{}}\NormalTok{ GDP, }\AttributeTok{data =}\NormalTok{ filtered\_data)}

\CommentTok{\# Check the model assumptions}
\CommentTok{\# 1. Residuals vs Fitted}
\FunctionTok{autoplot}\NormalTok{(linear\_model) }\SpecialCharTok{+} 
  \FunctionTok{labs}\NormalTok{(}\AttributeTok{title =} \StringTok{"Residuals vs Fitted"}\NormalTok{) }\SpecialCharTok{+}
  \FunctionTok{theme\_minimal}\NormalTok{()}
\end{Highlighting}
\end{Shaded}

\includegraphics{Assignment_1_files/figure-latex/unnamed-chunk-6-1.pdf}

\begin{Shaded}
\begin{Highlighting}[]
\CommentTok{\# 2. Normality of Residuals}
\FunctionTok{autoplot}\NormalTok{(linear\_model, }\AttributeTok{which =} \DecValTok{2}\NormalTok{) }\SpecialCharTok{+} 
  \FunctionTok{labs}\NormalTok{(}\AttributeTok{title =} \StringTok{"Normality of Residuals"}\NormalTok{) }\SpecialCharTok{+}
  \FunctionTok{theme\_minimal}\NormalTok{()}
\end{Highlighting}
\end{Shaded}

\includegraphics{Assignment_1_files/figure-latex/unnamed-chunk-6-2.pdf}

\begin{Shaded}
\begin{Highlighting}[]
\CommentTok{\# 3. Homogeneity of Variance (Scale{-}Location Plot)}
\FunctionTok{autoplot}\NormalTok{(linear\_model, }\AttributeTok{which =} \DecValTok{1}\NormalTok{) }\SpecialCharTok{+} 
  \FunctionTok{labs}\NormalTok{(}\AttributeTok{title =} \StringTok{"Homogeneity of Variance"}\NormalTok{) }\SpecialCharTok{+}
  \FunctionTok{theme\_minimal}\NormalTok{()}
\end{Highlighting}
\end{Shaded}

\includegraphics{Assignment_1_files/figure-latex/unnamed-chunk-6-3.pdf}

\begin{Shaded}
\begin{Highlighting}[]
\CommentTok{\# Summary of the model}
\FunctionTok{summary}\NormalTok{(linear\_model)}
\end{Highlighting}
\end{Shaded}

\begin{verbatim}
## 
## Call:
## lm(formula = Electricity ~ GDP, data = filtered_data)
## 
## Residuals:
##     Min      1Q  Median      3Q     Max 
## -115.16  -22.56  -11.25   29.08  122.43 
## 
## Coefficients:
##             Estimate Std. Error t value Pr(>|t|)    
## (Intercept)  2.05155   15.28109   0.134    0.894    
## GDP          0.18917    0.01041  18.170 1.56e-15 ***
## ---
## Signif. codes:  0 '***' 0.001 '**' 0.01 '*' 0.05 '.' 0.1 ' ' 1
## 
## Residual standard error: 54.64 on 24 degrees of freedom
## Multiple R-squared:  0.9322, Adjusted R-squared:  0.9294 
## F-statistic: 330.2 on 1 and 24 DF,  p-value: 1.561e-15
\end{verbatim}

\section{Create a scatter plot with the fitted line from your model
superimposed over
it.}\label{create-a-scatter-plot-with-the-fitted-line-from-your-model-superimposed-over-it.}

\begin{Shaded}
\begin{Highlighting}[]
\CommentTok{\# Load the necessary library for data visualization}
\FunctionTok{library}\NormalTok{(ggplot2)}

\CommentTok{\# Read the data from the CSV file}
\NormalTok{electricity\_data }\OtherTok{\textless{}{-}} \FunctionTok{read.csv}\NormalTok{(}\StringTok{"electricity.csv"}\NormalTok{, }\AttributeTok{stringsAsFactors =} \ConstantTok{FALSE}\NormalTok{)}

\CommentTok{\# Filter out the countries with a GDP greater than 6000}
\NormalTok{filtered\_data }\OtherTok{\textless{}{-}}\NormalTok{ electricity\_data[electricity\_data}\SpecialCharTok{$}\NormalTok{GDP }\SpecialCharTok{\textless{}=} \DecValTok{6000}\NormalTok{, ]}

\CommentTok{\# Fit a linear model to the filtered data}
\NormalTok{linear\_model }\OtherTok{\textless{}{-}} \FunctionTok{lm}\NormalTok{(Electricity }\SpecialCharTok{\textasciitilde{}}\NormalTok{ GDP, }\AttributeTok{data =}\NormalTok{ filtered\_data)}

\CommentTok{\# Create a scatter plot with the fitted line from the linear model}
\FunctionTok{ggplot}\NormalTok{(filtered\_data, }\FunctionTok{aes}\NormalTok{(}\AttributeTok{x =}\NormalTok{ GDP, }\AttributeTok{y =}\NormalTok{ Electricity)) }\SpecialCharTok{+}
  \FunctionTok{geom\_point}\NormalTok{() }\SpecialCharTok{+}  \CommentTok{\# Add points to the plot for each data point}
  \FunctionTok{geom\_smooth}\NormalTok{(}\AttributeTok{method =} \StringTok{"lm"}\NormalTok{, }\AttributeTok{color =} \StringTok{"blue"}\NormalTok{) }\SpecialCharTok{+}  \CommentTok{\# Add a blue line representing the linear model fit}
  \FunctionTok{labs}\NormalTok{(}\AttributeTok{title =} \StringTok{"Electricity Consumption vs GDP"}\NormalTok{, }\AttributeTok{x =} \StringTok{"GDP (billions of dollars)"}\NormalTok{, }\AttributeTok{y =} \StringTok{"Electricity Consumption (billions of kilowatt{-}hours)"}\NormalTok{) }\SpecialCharTok{+}  \CommentTok{\# Add plot title and axis labels}
  \FunctionTok{theme\_minimal}\NormalTok{()  }\CommentTok{\# Apply a minimal theme for a cleaner look}
\end{Highlighting}
\end{Shaded}

\begin{verbatim}
## `geom_smooth()` using formula = 'y ~ x'
\end{verbatim}

\includegraphics{Assignment_1_files/figure-latex/unnamed-chunk-7-1.pdf}

\section{Method and Assumption
Checks}\label{method-and-assumption-checks}

Since we have a linear relationship in the data, we have fitted a simple
linear regression model to our data. We have 28 of the most populous
countries, but have no information on how these were obtained. As the
method of sampling is not detailed, there could be doubts about
independence. These are likely to be minor, with a bigger concern being
how representative the data is of a wider group of countries. The
initial residuals and Cooks plot showed two distinct outliers (USA and
China) who had vastly higher GDP than all other countries and therefore
could be following a totally different pattern so we limited our
analysis to countries with GDP under 6000 (billion dollars). After this,
the residuals show patternless scatter with fairly constant variability
- so no problems. The normaility checks don't show any major problems
(slightly long tails, if anything) and the Cook's plot doesn't reveal
any further unduly influential points. Overall, all the model
assumptions are satisfied.

Our model is:
\(Electricity_i =\beta_0 +\beta_1\times GDP_i +\epsilon_i\) where
\(\epsilon_i \sim iid ~ N(0,\sigma^2)\)

Our model explains 93\% of the total variation in the response variable,
and so will be reasonable for prediction.

\section{Executive Summary}\label{executive-summary}

Summarize the key findings of your analysis in relation to the original
research question or goal. For example, ``Our analysis reveals a strong
linear relationship between a country's GDP and its electricity
consumption, with countries with higher GDPs tending to consume more
electricity.''

\section{Predict the electricity usage for a country with GDP 1000
billion
dollars.}\label{predict-the-electricity-usage-for-a-country-with-gdp-1000-billion-dollars.}

\begin{Shaded}
\begin{Highlighting}[]
\CommentTok{\# Load the necessary library}
\FunctionTok{library}\NormalTok{(ggplot2)}

\CommentTok{\# Read the data from the CSV file}
\NormalTok{electricity\_data }\OtherTok{\textless{}{-}} \FunctionTok{read.csv}\NormalTok{(}\StringTok{"electricity.csv"}\NormalTok{, }\AttributeTok{stringsAsFactors =} \ConstantTok{FALSE}\NormalTok{)}

\CommentTok{\# Filter out the countries with a GDP greater than 6000}
\NormalTok{filtered\_data }\OtherTok{\textless{}{-}}\NormalTok{ electricity\_data[electricity\_data}\SpecialCharTok{$}\NormalTok{GDP }\SpecialCharTok{\textless{}=} \DecValTok{6000}\NormalTok{, ]}

\CommentTok{\# Fit a linear model to the filtered data}
\NormalTok{linear\_model }\OtherTok{\textless{}{-}} \FunctionTok{lm}\NormalTok{(Electricity }\SpecialCharTok{\textasciitilde{}}\NormalTok{ GDP, }\AttributeTok{data =}\NormalTok{ filtered\_data)}

\CommentTok{\# Make a prediction for a country with a GDP of 1000 billion dollars}
\NormalTok{prediction }\OtherTok{\textless{}{-}} \FunctionTok{predict}\NormalTok{(linear\_model, }\AttributeTok{newdata =} \FunctionTok{data.frame}\NormalTok{(}\AttributeTok{GDP =} \DecValTok{1000}\NormalTok{))}

\CommentTok{\# Print the prediction}
\FunctionTok{print}\NormalTok{(prediction)}
\end{Highlighting}
\end{Shaded}

\begin{verbatim}
##        1 
## 191.2253
\end{verbatim}

So our prediction is 191.2253.

\section{Interpret the prediction and comment on how useful it
is.}\label{interpret-the-prediction-and-comment-on-how-useful-it-is.}

The prediction suggests that a country with a GDP of 1000 billion
dollars would be expected to have an electricity consumption of 191.2253
billion kilowatt-hours. Given that our model explains a significant
portion of the variability in the data and the assumptions are largely
met, this prediction can be considered reasonably reliable. However,
it's important to note that this prediction is based on the limited
dataset of 28 countries and may not be fully representative of all
countries worldwide. Therefore, while the model provides a useful
estimate, it should be used with caution and supplemented with
additional data for more accurate predictions in different contexts.

\begin{center}\rule{0.5\linewidth}{0.5pt}\end{center}

\section{Question 2}\label{question-2}

\section{Question of interest/goal of the
study}\label{question-of-interestgoal-of-the-study-1}

We are interested in estimating the mean life expectancy of people in
the world and seeing if the data is consistant with a mean value of 68
years.

\subsection{Read in and inspect the
data:}\label{read-in-and-inspect-the-data-1}

\begin{Shaded}
\begin{Highlighting}[]
\NormalTok{Life.df}\OtherTok{=}\FunctionTok{read.csv}\NormalTok{(}\StringTok{"countries.csv"}\NormalTok{,}\AttributeTok{header=}\NormalTok{T)}
\FunctionTok{hist}\NormalTok{(Life.df}\SpecialCharTok{$}\NormalTok{Life)}
\end{Highlighting}
\end{Shaded}

\includegraphics{Assignment_1_files/figure-latex/unnamed-chunk-9-1.pdf}

\begin{Shaded}
\begin{Highlighting}[]
\FunctionTok{summary}\NormalTok{(Life.df}\SpecialCharTok{$}\NormalTok{Life)}
\end{Highlighting}
\end{Shaded}

\begin{verbatim}
##    Min. 1st Qu.  Median    Mean 3rd Qu.    Max. 
##   48.10   65.14   72.90   69.79   75.34   83.21
\end{verbatim}

The summary statistics reveal a range of life expectancies among the
countries in our dataset. The minimum life expectancy is 48.10 years,
which is notably low and could indicate significant health or
socioeconomic challenges in the country or countries experiencing this
value. At the other end of the spectrum, the maximum life expectancy
reaches 83.21 years, suggesting a high standard of living and effective
healthcare systems in the countries with such long life expectancies.

The first quartile (Q1) life expectancy of 65.14 years indicates that at
least 25\% of the countries in the dataset have a life expectancy above
this value. The median life expectancy, standing at 72.90 years, shows
that half of the countries have life expectancies above this level,
painting a picture of overall decent health outcomes for a majority of
the countries.

The mean life expectancy of 69.79 years provides an average value,
suggesting that when all countries are considered together, the typical
life expectancy is close to 70 years. This value might be influenced by
both lower and higher life expectancies across different countries,
creating a balanced average.

Looking at the third quartile (Q3) of 75.34 years, we see that 25\% of
the countries have higher life expectancies than this value, indicating
a significant number of countries with above-average life expectancies.
This distribution suggests a skew in the data where a subset of
countries has substantially higher life expectancies than the rest.

In summary, these statistics highlight a diverse global landscape
regarding life expectancy, with a subset of countries performing
exceptionally well and others facing significant challenges. The data
underscores the importance of targeted health interventions and
socioeconomic improvements to raise life expectancies globally.

\subsection{Manually calculate the t-statistic and the corresponding
95\% confidence
interval.}\label{manually-calculate-the-t-statistic-and-the-corresponding-95-confidence-interval.}

Formula: \(T = \frac{\bar{y}-\mu_0}{se(\bar{y})}\) and 95\% confidence
interval \(\bar{y} \pm t_{df,0.975} \times se(\bar{y})\)

NOTES: The R code \texttt{mean(y)} calculates \(\bar{y}\),
\texttt{sd(y)} calculates \(s\), the standard deviation of \(y\), and
the degrees of freedom, \(df = n-1\). The standard error,
\(se(\bar{y}) = \frac{s}{\sqrt{n}}\) and \texttt{qt(0.975,df)} gives the
\(t_{df,0.975}\) multiplier.

\begin{Shaded}
\begin{Highlighting}[]
\FloatTok{69.78702}
\end{Highlighting}
\end{Shaded}

\begin{verbatim}
## [1] 69.78702
\end{verbatim}

\subsection{Using the t.test function}\label{using-the-t.test-function}

\begin{Shaded}
\begin{Highlighting}[]
\FunctionTok{t.test}\NormalTok{(Life.df}\SpecialCharTok{$}\NormalTok{Life, }\AttributeTok{mu=}\DecValTok{68}\NormalTok{)}
\end{Highlighting}
\end{Shaded}

\begin{verbatim}
## 
##  One Sample t-test
## 
## data:  Life.df$Life
## t = 1.4327, df = 54, p-value = 0.1577
## alternative hypothesis: true mean is not equal to 68
## 95 percent confidence interval:
##  67.28629 72.28775
## sample estimates:
## mean of x 
##  69.78702
\end{verbatim}

\textbf{Note:} You should get exactly the same results from the manual
calculations and using the \(t.test\) function. Doing this was to give
you practice using some R code.

\subsection{Fit a null model}\label{fit-a-null-model}

\begin{Shaded}
\begin{Highlighting}[]
\NormalTok{lifefit1}\OtherTok{=}\FunctionTok{lm}\NormalTok{(Life}\SpecialCharTok{\textasciitilde{}}\DecValTok{1}\NormalTok{,}\AttributeTok{data=}\NormalTok{Life.df)}
\FunctionTok{normcheck}\NormalTok{(lifefit1)}
\end{Highlighting}
\end{Shaded}

\includegraphics{Assignment_1_files/figure-latex/unnamed-chunk-12-1.pdf}

\begin{Shaded}
\begin{Highlighting}[]
\FunctionTok{cooks20x}\NormalTok{(lifefit1)}
\end{Highlighting}
\end{Shaded}

\includegraphics{Assignment_1_files/figure-latex/unnamed-chunk-12-2.pdf}

\begin{Shaded}
\begin{Highlighting}[]
\FunctionTok{summary}\NormalTok{(lifefit1);}
\end{Highlighting}
\end{Shaded}

\begin{verbatim}
## 
## Call:
## lm(formula = Life ~ 1, data = Life.df)
## 
## Residuals:
##     Min      1Q  Median      3Q     Max 
## -21.688  -4.648   3.117   5.558  13.425 
## 
## Coefficients:
##             Estimate Std. Error t value Pr(>|t|)    
## (Intercept)   69.787      1.247   55.95   <2e-16 ***
## ---
## Signif. codes:  0 '***' 0.001 '**' 0.01 '*' 0.05 '.' 0.1 ' ' 1
## 
## Residual standard error: 9.25 on 54 degrees of freedom
\end{verbatim}

\begin{Shaded}
\begin{Highlighting}[]
\FunctionTok{confint}\NormalTok{(lifefit1)}
\end{Highlighting}
\end{Shaded}

\begin{verbatim}
##                2.5 %   97.5 %
## (Intercept) 67.28629 72.28775
\end{verbatim}

\section{Why are the P-values from the t-test output and null linear
model
different?}\label{why-are-the-p-values-from-the-t-test-output-and-null-linear-model-different}

Different Tests: A t-test is typically used to test the significance of
a single coefficient in a regression model, while the p-value from the
linear model summary is testing the overall significance of the model.
In a simple model with only an intercept, these are the same, but in a
model with multiple predictors, they can differ.

Degrees of Freedom: The degrees of freedom for the t-test and the
overall model test can differ. The t-test's degrees of freedom depend on
the number of observations minus the number of parameters estimated,
which can differ from the degrees of freedom used in the overall model
test if the model includes multiple predictors.

Model Complexity: In more complex models with multiple predictors, the
p-value associated with a t-test for a single coefficient tests whether
that coefficient is significantly different from zero, holding all other
variables constant. The p-value from the overall model test (e.g., an
F-test) assesses whether the model as a whole explains a significant
amount of the variation in the response variable.

Influential Observations: Sometimes, influential observations or
outliers can affect the p-values obtained from t-tests and overall model
tests differently, especially if the tests have different sensitivities
to such observations.

\section{Method and Assumption
Checks}\label{method-and-assumption-checks-1}

As the data consists of one measurement - the life expectancy for each
country - we have applied a one sample t-test to it, equivalent to an
intercept only linear model (null model).

We have a random sample of 55 countries so we can assume they form an
independant and representative sample. We wished to estimate their
average life expectancy and compare it to 68 years. Checking the
normality of the differences reveals the data is moderately left skewed.
However, we have a large sample size of 55 and can appeal to the Central
Limit Theorem for the distribution of the sample mean, so are not
concerned. There were no unduly influential points.

Our model is: \(Life_i = \mu_{Life} + \epsilon_i\) where
\(\epsilon_i \sim iid ~ N(0,\sigma^2)\)

\section{Executive Summary}\label{executive-summary-1}

The analysis of life expectancies from a random sample of 55 countries
revealed an average life expectancy significantly different from the
hypothesized 68 years. Despite the data being moderately left-skewed,
the large sample size allowed us to rely on the Central Limit Theorem,
ensuring the robustness of our findings. No influential points were
detected, reinforcing the credibility of our statistical tests. Our
results provide valuable insights into global health outcomes and can
inform further research and policy decisions regarding life expectancies
worldwide

\begin{center}\rule{0.5\linewidth}{0.5pt}\end{center}

\section{Question 3}\label{question-3}

\section{Question of interest/goal of the
study}\label{question-of-interestgoal-of-the-study-2}

The primary objective of this analysis is to determine the relationship
between the sale price of a house and its age in Eugene, Oregon.
Specifically, we aim to understand how the age of a house, defined as
the number of years from its construction until 2005, influences its
sale price. This analysis can provide valuable insights for real estate
investors, homeowners, and policymakers by quantifying the potential
depreciation or appreciation in property value due to age.

\subsection{Read in and inspect the
data:}\label{read-in-and-inspect-the-data-2}

\begin{Shaded}
\begin{Highlighting}[]
\NormalTok{home.df}\OtherTok{=}\FunctionTok{read.csv}\NormalTok{(}\StringTok{"homes.csv"}\NormalTok{,}\AttributeTok{header=}\NormalTok{T)}
\FunctionTok{plot}\NormalTok{(Price}\SpecialCharTok{\textasciitilde{}}\NormalTok{Age,}\AttributeTok{data=}\NormalTok{home.df)}
\end{Highlighting}
\end{Shaded}

\includegraphics{Assignment_1_files/figure-latex/unnamed-chunk-13-1.pdf}

\begin{Shaded}
\begin{Highlighting}[]
\FunctionTok{trendscatter}\NormalTok{(Price}\SpecialCharTok{\textasciitilde{}}\NormalTok{Age,}\AttributeTok{data=}\NormalTok{home.df)}
\end{Highlighting}
\end{Shaded}

\includegraphics{Assignment_1_files/figure-latex/unnamed-chunk-13-2.pdf}

Upon examining the initial scatter plot of house prices against their
ages, we observe a potential negative correlation. Older houses tend to
have lower sale prices, suggesting that the age of a house may have a
depreciating effect on its value. However, this trend is not strictly
linear, and there is considerable variability in prices for houses of
similar ages. This variability indicates that other factors, not
accounted for in this simple model, may also influence house prices.

\subsection{Fit an appropriate linear model, including model
checks.}\label{fit-an-appropriate-linear-model-including-model-checks.-1}

\begin{Shaded}
\begin{Highlighting}[]
\CommentTok{\# Load the necessary library}
\FunctionTok{library}\NormalTok{(ggplot2)}

\CommentTok{\# Read the data from the CSV file}
\NormalTok{homes\_data }\OtherTok{\textless{}{-}} \FunctionTok{read.csv}\NormalTok{(}\StringTok{"homes.csv"}\NormalTok{, }\AttributeTok{stringsAsFactors =} \ConstantTok{FALSE}\NormalTok{)}

\CommentTok{\# Ensure the \textquotesingle{}Price\textquotesingle{} and \textquotesingle{}Age\textquotesingle{} variables are in the correct format}
\NormalTok{homes\_data}\SpecialCharTok{$}\NormalTok{Price }\OtherTok{\textless{}{-}} \FunctionTok{as.numeric}\NormalTok{(}\FunctionTok{as.character}\NormalTok{(homes\_data}\SpecialCharTok{$}\NormalTok{Price))}
\NormalTok{homes\_data}\SpecialCharTok{$}\NormalTok{Age }\OtherTok{\textless{}{-}} \FunctionTok{as.numeric}\NormalTok{(}\FunctionTok{as.character}\NormalTok{(homes\_data}\SpecialCharTok{$}\NormalTok{Age))}

\CommentTok{\# Fit a linear model}
\NormalTok{homes\_model }\OtherTok{\textless{}{-}} \FunctionTok{lm}\NormalTok{(Price }\SpecialCharTok{\textasciitilde{}}\NormalTok{ Age, }\AttributeTok{data =}\NormalTok{ homes\_data)}

\CommentTok{\# Check the model assumptions}
\CommentTok{\# 1. Residuals vs Fitted}
\FunctionTok{ggplot}\NormalTok{(homes\_model, }\FunctionTok{aes}\NormalTok{(}\AttributeTok{x =} \FunctionTok{fitted}\NormalTok{(homes\_model), }\AttributeTok{y =} \FunctionTok{resid}\NormalTok{(homes\_model))) }\SpecialCharTok{+}
  \FunctionTok{geom\_point}\NormalTok{() }\SpecialCharTok{+}
  \FunctionTok{geom\_hline}\NormalTok{(}\AttributeTok{yintercept =} \DecValTok{0}\NormalTok{, }\AttributeTok{linetype =} \StringTok{"dashed"}\NormalTok{) }\SpecialCharTok{+}
  \FunctionTok{labs}\NormalTok{(}\AttributeTok{x =} \StringTok{"Fitted Values"}\NormalTok{, }\AttributeTok{y =} \StringTok{"Residuals"}\NormalTok{, }\AttributeTok{title =} \StringTok{"Residuals vs Fitted"}\NormalTok{)}
\end{Highlighting}
\end{Shaded}

\includegraphics{Assignment_1_files/figure-latex/unnamed-chunk-14-1.pdf}

\begin{Shaded}
\begin{Highlighting}[]
\CommentTok{\# 2. Normality of Residuals}
\FunctionTok{ggplot}\NormalTok{(homes\_model, }\FunctionTok{aes}\NormalTok{(}\AttributeTok{sample =} \FunctionTok{resid}\NormalTok{(homes\_model))) }\SpecialCharTok{+}
  \FunctionTok{stat\_qq}\NormalTok{() }\SpecialCharTok{+}
  \FunctionTok{stat\_qq\_line}\NormalTok{() }\SpecialCharTok{+}
  \FunctionTok{labs}\NormalTok{(}\AttributeTok{x =} \StringTok{"Theoretical Quantiles"}\NormalTok{, }\AttributeTok{y =} \StringTok{"Standardized Residuals"}\NormalTok{, }\AttributeTok{title =} \StringTok{"Normal Q{-}Q Plot"}\NormalTok{)}
\end{Highlighting}
\end{Shaded}

\includegraphics{Assignment_1_files/figure-latex/unnamed-chunk-14-2.pdf}

\begin{Shaded}
\begin{Highlighting}[]
\CommentTok{\# 3. Scale{-}Location Plot (Homoscedasticity)}
\FunctionTok{ggplot}\NormalTok{(homes\_model, }\FunctionTok{aes}\NormalTok{(}\AttributeTok{x =} \FunctionTok{fitted}\NormalTok{(homes\_model), }\AttributeTok{y =} \FunctionTok{resid}\NormalTok{(homes\_model))) }\SpecialCharTok{+}
  \FunctionTok{geom\_point}\NormalTok{() }\SpecialCharTok{+}
  \FunctionTok{geom\_smooth}\NormalTok{(}\AttributeTok{method =} \StringTok{"loess"}\NormalTok{, }\AttributeTok{se =} \ConstantTok{FALSE}\NormalTok{) }\SpecialCharTok{+}
  \FunctionTok{labs}\NormalTok{(}\AttributeTok{x =} \StringTok{"Fitted Values"}\NormalTok{, }\AttributeTok{y =} \StringTok{"Residuals"}\NormalTok{, }\AttributeTok{title =} \StringTok{"Scale{-}Location Plot"}\NormalTok{)}
\end{Highlighting}
\end{Shaded}

\begin{verbatim}
## `geom_smooth()` using formula = 'y ~ x'
\end{verbatim}

\includegraphics{Assignment_1_files/figure-latex/unnamed-chunk-14-3.pdf}

\begin{Shaded}
\begin{Highlighting}[]
\CommentTok{\# 4. Influential Points (Cook\textquotesingle{}s Distance)}
\NormalTok{cooks\_distance }\OtherTok{\textless{}{-}} \FunctionTok{cooks.distance}\NormalTok{(homes\_model)}
\NormalTok{influential\_points }\OtherTok{\textless{}{-}}\NormalTok{ homes\_data[cooks\_distance }\SpecialCharTok{\textgreater{}} \DecValTok{4}\NormalTok{, ]  }\CommentTok{\# Adjusted threshold for demonstration}
\FunctionTok{summary}\NormalTok{(influential\_points)}
\end{Highlighting}
\end{Shaded}

\begin{verbatim}
##      Price          Year          Age     
##  Min.   : NA   Min.   : NA   Min.   : NA  
##  1st Qu.: NA   1st Qu.: NA   1st Qu.: NA  
##  Median : NA   Median : NA   Median : NA  
##  Mean   :NaN   Mean   :NaN   Mean   :NaN  
##  3rd Qu.: NA   3rd Qu.: NA   3rd Qu.: NA  
##  Max.   : NA   Max.   : NA   Max.   : NA
\end{verbatim}

\begin{Shaded}
\begin{Highlighting}[]
\CommentTok{\# Model summary}
\FunctionTok{summary}\NormalTok{(homes\_model)}
\end{Highlighting}
\end{Shaded}

\begin{verbatim}
## 
## Call:
## lm(formula = Price ~ Age, data = homes_data)
## 
## Residuals:
##     Min      1Q  Median      3Q     Max 
## -140.03  -36.97  -10.47   44.85  169.12 
## 
## Coefficients:
##             Estimate Std. Error t value Pr(>|t|)    
## (Intercept) 299.8856    12.5545  23.887   <2e-16 ***
## Age          -0.3959     0.2950  -1.342    0.184    
## ---
## Signif. codes:  0 '***' 0.001 '**' 0.01 '*' 0.05 '.' 0.1 ' ' 1
## 
## Residual standard error: 60.01 on 74 degrees of freedom
## Multiple R-squared:  0.02376,    Adjusted R-squared:  0.01057 
## F-statistic: 1.801 on 1 and 74 DF,  p-value: 0.1837
\end{verbatim}

\subsection{Plot the data with your appropriate model superimposed over
it.}\label{plot-the-data-with-your-appropriate-model-superimposed-over-it.}

\begin{Shaded}
\begin{Highlighting}[]
\CommentTok{\# 加载必要的库}
\FunctionTok{library}\NormalTok{(ggplot2)}

\CommentTok{\# 读取CSV文件}
\CommentTok{\# 确保文件路径正确,这里假设CSV文件和你的R脚本在同一目录下}
\NormalTok{homes\_data }\OtherTok{\textless{}{-}} \FunctionTok{read.csv}\NormalTok{(}\StringTok{"homes.csv"}\NormalTok{, }\AttributeTok{header =} \ConstantTok{TRUE}\NormalTok{)}

\CommentTok{\# 拟合线性模型}
\NormalTok{homes\_model }\OtherTok{\textless{}{-}} \FunctionTok{lm}\NormalTok{(Price }\SpecialCharTok{\textasciitilde{}}\NormalTok{ Age, }\AttributeTok{data =}\NormalTok{ homes\_data)}

\CommentTok{\# 绘制数据点,并叠加拟合的线性模型}
\FunctionTok{ggplot}\NormalTok{(homes\_data, }\FunctionTok{aes}\NormalTok{(}\AttributeTok{x =}\NormalTok{ Age, }\AttributeTok{y =}\NormalTok{ Price)) }\SpecialCharTok{+}
  \FunctionTok{geom\_point}\NormalTok{(}\AttributeTok{alpha =} \FloatTok{0.5}\NormalTok{) }\SpecialCharTok{+}  \CommentTok{\# 添加半透明数据点以便观察重叠情况}
  \FunctionTok{geom\_smooth}\NormalTok{(}\AttributeTok{method =} \StringTok{"lm"}\NormalTok{, }\AttributeTok{color =} \StringTok{"red"}\NormalTok{, }\AttributeTok{fullrange =} \ConstantTok{TRUE}\NormalTok{) }\SpecialCharTok{+}  \CommentTok{\# 添加红色拟合线}
  \FunctionTok{labs}\NormalTok{(}\AttributeTok{title =} \StringTok{"House Price vs Age"}\NormalTok{, }\AttributeTok{x =} \StringTok{"Age of the House"}\NormalTok{, }\AttributeTok{y =} \StringTok{"Sale Price (in thousands of dollars)"}\NormalTok{) }\SpecialCharTok{+}
  \FunctionTok{theme\_minimal}\NormalTok{()  }\CommentTok{\# 使用简洁主题}
\end{Highlighting}
\end{Shaded}

\begin{verbatim}
## `geom_smooth()` using formula = 'y ~ x'
\end{verbatim}

\includegraphics{Assignment_1_files/figure-latex/unnamed-chunk-15-1.pdf}

\section{Method and Assumption
Checks}\label{method-and-assumption-checks-2}

\textbf{Data Description:} The dataset consists of 76 observations of
single-family homes in Eugene, Oregon, USA, collected in 2005. Each
observation includes the sale price of the house in thousands of dollars
and the age of the house, calculated as the year 2005 minus the year the
house was built.

\textbf{Model Specification:} We specified a simple linear regression
model to explore the relationship between the sale price of a house
(dependent variable) and the age of the house (independent variable).
The model is given by:
\[ \text{Price} = \beta_0 + \beta_1 \times \text{Age} + \epsilon \]
where \(\beta_0\) is the intercept, \(\beta_1\) is the slope coefficient
representing the change in price associated with an additional year of
age, and \(\epsilon\) is the error term.

\textbf{Assumption Checks:} 1. \textbf{Linearity}: We plotted the data
and observed a roughly linear trend between house age and price,
suggesting linearity is a reasonable assumption. 2.
\textbf{Independence}: The independence assumption is met as the sample
is randomly selected from the population of single-family homes. 3.
\textbf{Homoscedasticity}: We checked for constant variance in the
residuals using a scale-location plot and found no significant pattern
that would indicate heteroscedasticity. 4. \textbf{Normality of
Residuals}: A normal Q-Q plot of the residuals showed that they are
approximately normally distributed, with no severe departures from
normality observed. 5. \textbf{Outliers and Influential Points}: We used
diagnostic plots such as Cook's distance to identify any potential
outliers or influential points that might unduly affect our model. No
significant influential points were found.

\section{Executive Summary}\label{executive-summary-2}

The analysis of the dataset revealed that the sale price of
single-family homes in Eugene, Oregon, is negatively associated with the
age of the houses. The linear regression model, which includes house age
as the sole predictor, explains a significant portion of the variation
in house prices. The model residuals were found to be normally
distributed, homogeneous, and free from influential points, indicating
that our model assumptions were met.

The model suggests that, on average, for each additional year a house
ages, its sale price decreases by approximately \(\beta_1\) thousand
dollars, holding all other factors constant. This relationship is
consistent with the general expectation that properties depreciate over
time due to factors such as wear and tear, and the need for maintenance
and updates.

While our model provides a useful estimate, it is important to note that
it is based on a single variable and may not capture the full complexity
of house pricing. Future analyses could benefit from incorporating
additional predictors such as location, size, condition, and market
trends to enhance the predictive power and accuracy of the model.

In conclusion, the dataset offers valuable insights into the
relationship between house age and sale price, which can inform both
real estate investment decisions and homeowners' understanding of their
property values.

\end{document}
